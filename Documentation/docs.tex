% !TEX encoding = UTF-8 Unicode
\documentclass[14pt,a4paper]{article}

\usepackage{lipsum}
\usepackage[margin=2cm,includefoot]{geometry}
\usepackage[utf8]{inputenc}
\usepackage[russian]{babel}
\usepackage{amssymb}
\usepackage{amsmath}
\usepackage{amsthm}
\usepackage{latexsym}
\usepackage{dsfont}
\usepackage[linesnumbered]{algorithm2e}
\usepackage{mathtools}

\usepackage{tikz} 
\usepackage{tikz-qtree}

\usepackage{graphicx}
\usepackage{float}
\graphicspath{ {./} }

\usepackage{tabularx}
\usepackage{makecell}
\usepackage{multirow}

\newcolumntype{q}{>{\hsize=.4\hsize}X}
\newcolumntype{w}{>{\hsize=1.35\hsize}X}
\newcolumntype{e}{>{\hsize=.25\hsize}X}

\usepackage{titling}
\renewcommand\maketitlehooka{\null\mbox{}\vfill}
\renewcommand\maketitlehookd{\vfill\null}

%\DeclarePairedDelimiter\ceil{\lceil}{\rceil}
%\DeclarePairedDelimiter\floor{\lfloor}{\rfloor}

%\def\changemargin#1#2{\list{}{\rightmargin#2\leftmargin#1}\item[]}
%\let\endchangemargin=\endlist 

%\newcommand\tab[1][0.7cm]{\hspace*{#1}}

%\newcommand{\hm}[1]{#1\nobreak\discretionary{}{\hbox{\ensuremath{#1}}}{}}

%\renewcommand*{\qed}{\hfill\ensuremath{\blacksquare}}%

\usepackage{fancyhdr}
\pagestyle{fancy}
\thispagestyle{empty}
\fancyhead{}
\fancyfoot{}
\fancyfoot[R]{ \thepage\ }
\renewcommand{\headrulewidth}{0pt}
\renewcommand{\footrulewidth}{1pt}

\usepackage{hyperref}
\hypersetup{
    colorlinks=false, %set true if you want colored links
    linktoc=all,     %set to all if you want both sections and subsections linked
    %linkcolor=blue,  %choose some color if you want links to stand out
}

\begin{document}

\text{}
\vskip 8cm
\begin{center}
\begin{minipage}{0.8\textwidth}
\begin{center}
\Huge Веб-приложение для распознавания музыкальных инструментов
\end{center}
\end{minipage}
\end{center}

\vskip 7cm

\begin{flushright}
\Large \underline{Авторы}: \\
Бухараев Алим \\
Прохоров Юрий \\
Савелов Михаил \\
Яушев Фарух
\end{flushright}

\vskip 3.6cm

\begin{center}
2019
\end{center}

\newpage

\renewcommand\contentsname{\huge Содержание}
\setcounter{tocdepth}{2}
\Large \tableofcontents
\normalsize

\newpage 

\section[Описание задачи]{\huge Описание задачи}

\subsection{Аннотация}

\qquad Целью проекта является, в первую очередь, повышение командных навыков вышеперечисленных студентов второго курса ФУПМа (далее — мы). Нам в голову пришла мысль написать веб-приложение, способное распознавать различные музыкальные инструменты на загруженной пользователем аудиозаписи. Как выяснилось, данная задача является открытой проблемой в машинном обучении. Многие ведущие учёные приложили руку к её решению, тем не менее точных результатов до сих пор получено не было. Мы решили попробовать, если не решить эту проблему, то хотя бы помочь мировому сообществу в увеличении размера датасета для дальнейшего обучения. \\

Пользователю предлагается возможность записать свой инструмент на диктофон или же загрузить готовый файл на сервер. После этого нейронная сеть пытается предсказать, что за инструмент звучит на аудиозаписи. Пользователь отправит сообщение о том, угадала ли сеть, а аудиозапись может быть использована далее для дальнешего обучения нашей или каких-либо других моделей. \\

Здесь будет заключительный абзац.

\subsection{Техническое задание}

\begin{table}[H]
\begin{tabularx}{\textwidth}{qwe}
\hline
\multicolumn{1}{|c|}{Член команды} & \multicolumn{1}{c|}{Задача} & \multicolumn{1}{c|}{Срок} \\ \hline
 &  &  \\ \hline
\multicolumn{1}{|c|}{\multirow{3}{*}{Бухараев Алим}} & \multicolumn{1}{c|}{Изучение сверточных и рекуррентных нейронных сетей} & \multicolumn{1}{c|}{} \\ \cline{2-3} 
\multicolumn{1}{|c|}{} & \multicolumn{1}{c|}{Составление модели-классификатора} & \multicolumn{1}{c|}{} \\ \cline{2-3} 
\multicolumn{1}{|c|}{} & \multicolumn{1}{c|}{} & \multicolumn{1}{c|}{} \\ \hline
 &  &  \\ \hline
\multicolumn{1}{|c|}{\multirow{6}{*}{Прохоров Юрий}} & \multicolumn{1}{c|}{Реализация анализатора файла формата .WAV} & \multicolumn{1}{c|}{20.02.19} \\ \cline{2-3} 
\multicolumn{1}{|c|}{} & \multicolumn{1}{c|}{\makecell{Спектральный анализ аудиозаписей с помощью дискретного \\ преобразования Фурье}} & \multicolumn{1}{c|}{} \\ \cline{2-3} 
\multicolumn{1}{|c|}{} & \multicolumn{1}{c|}{} & \multicolumn{1}{c|}{} \\ \cline{2-3} 
\multicolumn{1}{|c|}{} & \multicolumn{1}{c|}{} & \multicolumn{1}{c|}{} \\ \cline{2-3} 
\multicolumn{1}{|c|}{} & \multicolumn{1}{c|}{} & \multicolumn{1}{c|}{} \\ \hline
 &  &  \\ \hline
\multicolumn{1}{|c|}{\multirow{3}{*}{Савелов Михаил}} & \multicolumn{1}{c|}{\makecell{Поиск большого числа мелодий, пригодных для использования \\ в качестве обучающей выборки}} & \multicolumn{1}{c|}{} \\ \cline{2-3} 
\multicolumn{1}{|c|}{} & \multicolumn{1}{c|}{} & \multicolumn{1}{c|}{} \\ \cline{2-3} 
\multicolumn{1}{|c|}{} & \multicolumn{1}{c|}{} & \multicolumn{1}{c|}{} \\ \hline
 &  &  \\ \hline
\multicolumn{1}{|c|}{\multirow{5}{*}{Яушев Фарух}} & \multicolumn{1}{c|}{Изучение языков HTML5, JavaScript и CSS} & \multicolumn{1}{c|}{} \\ \cline{2-3} 
\multicolumn{1}{|c|}{} & \multicolumn{1}{c|}{Моделирование пользовательского интерфейса} & \multicolumn{1}{c|}{} \\ \cline{2-3} 
\multicolumn{1}{|c|}{} & \multicolumn{1}{c|}{} & \multicolumn{1}{c|}{} \\ \cline{2-3} 
\multicolumn{1}{|c|}{} & \multicolumn{1}{c|}{} & \multicolumn{1}{c|}{} \\ \cline{2-3} 
\multicolumn{1}{|c|}{} & \multicolumn{1}{c|}{} & \multicolumn{1}{c|}{} \\ \hline
\end{tabularx}
\end{table}

\underline{Клиентская часть:}
\begin{enumerate}
\item Придумать функционал и скелет веб-страницы. \\
\qquad --- библиотека jquery
\item Реализовать этот скелет. \\
\qquad --- заготовки CSS структур есть на сайте codepen.io
\item Связать внешний микрофон с кнопкой (возможно преобразовать в .WAV) \\
\qquad --- либо сразу загружать .WAV файл
\item Интерактивная графика (библиотека SVG, ...)
\end{enumerate}
\vskip 0.7cm
\noindent \underline{Серверная часть:}
\begin{enumerate}
\item Распарсить .WAV файл  \\
\qquad --- возможно научиться парсить MP3
\item Преобразовать к нужному виду \\
\qquad --- FFT \\
\qquad --- выделить пики \\
\qquad --- ... \\
\qquad --- спектрограмма (для визуализации)
\item Понять, какой формат ввода будет у нейронной сети. 
\item Написать нейронную сеть \\
\qquad --- библиотека TensorFlow 
\item Попробовать решить задачу другими методами. \\
\qquad --- случайный лес \\
\qquad --- ...
\item Собрать данные \\
\qquad --- найти чистые записи музыкальных инструментов \\
\qquad --- нарезать эти видео
\end{enumerate}
\vskip 0.7cm
\noindent \underline{Промежуточная часть:}
\begin{enumerate}
\item Разобраться в клиент-серверном приложении: \\
\qquad --- Remote Procedure Call
\item ...
\end{enumerate}
\subsection{Используемые средства}

\newpage

\section[Методика]{\huge Методика}
\subsection{Обработка цифровых сигналов}
\subsection{Подготовка обучаемой выборки}
\subsection{Модель распознавания}
\subsection{Реализация пользовательского интерфейса}
\subsection{Серверная часть}

\newpage

\section[Документация]{\huge Документация}
\subsection{Описание файлов}
\subsection{Конкретные файлы}

\newpage

\section[Ссылки]{\huge Ссылки}

\end{document}



